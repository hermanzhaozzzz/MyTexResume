\documentclass[11pt]{article} % Choose the document class and font size
\usepackage[left=0.6in,top=0.6in,right=0.8in,bottom=0.6in]{geometry} 
% \usepackage{titlesec}
% \titleformat{\section}{\normalfont\Large\bfseries}{\thesection}{1em}{}[\hrule]



% Set the citation style
\usepackage[
    backend=bibtex,      % Specifies the backend to be used by BibLaTeX for processing the bibliography. 'biber' is the default backend.
    maxnames=20,        % Limits the maximum number of author names to display before abbreviating with "et al."
    style=nature,       % Sets the citation style to 'nature,' which is commonly used in scientific papers.
    sorting=none,       % Specifies the sorting order of entries in the bibliography:
                        % y - year (descending)
                        % d - descending order
                        % n - name
                        % t - title
                        % none - follow the order in ref.bib
    % defernumbers=true,  % Delays the assignment of citation numbers until the end of the document, allowing for the correct order of citations within each bibliography section.
]{biblatex}
\addbibresource{ref.bib} % Adds the bibliography resource file 'ref.bib' containing all the references.

% Allows columns that stretch across pages
\usepackage{longtable}

% Table functionality and beautification (not strictly needed)
\usepackage{bookmark}

% Use icons, if you want.
% All available icons: http://mirrors.ibiblio.org/CTAN/fonts/fontawesome5/doc/fontawesome5.pdf
\usepackage{fontawesome}

% Allows font justification control (needed for clean pub-list formatting)
\usepackage{ragged2e}

% For underlining with line breaks
\usepackage{soul} 

% All fonts: https://tug.org/FontCatalogue/
\usepackage{kpfonts} % More professional font
% \usepackage[default]{sourcecodepro} % Code-like font
\usepackage[T1]{fontenc}

% Control hyperlinks and colors
% CUSTOM COLORS INCLUDED DIRECTLY AFTER \begin{document}
\usepackage{xcolor}
\usepackage{hyperref}
\hypersetup{
    colorlinks=true,        % Enable colored links
    breaklinks=true,        % Allow links to break across lines
    linkcolor=cornflowerblue,    % Color of internal links
    urlcolor=cornflowerblue,     % Color of URL links
    anchorcolor=cornflowerblue,  % Color of anchors
    citecolor=cornflowerblue,    % Color of citations
    pdftitle={Your Title},    % Title of the PDF
    pdfauthor={Your Name}, % Author of the PDF
    bookmarksopen=true,      % Open bookmarks panel at start
}

%%% CONVENIENCE FUNCTIONS GO HERE %%%
%%% ----------------------------- %%%
\newcommand{\mytitle}[1]{
  \begin{center}
    Curriculum Vitae\\
    \hspace{4pt} \\ % Name in large bold font
    \Large\textbf{#1}\normalsize \\
    \hspace{4pt} \\ % Name in large bold font
  \end{center}
}
%%% ----------------------------- %%%


\begin{document}
% Set custom colors here (imported directly after \begin{document})
% The below use HTML hex codes.
% More HTML hex codes: https://encycolorpedia.com/html
\definecolor{firebrick}{HTML}{b22222} 
\definecolor{darkslategrey}{HTML}{2f4f4f} 
\definecolor{cornflowerblue}{HTML}{6495ed} 
\definecolor{mediumslateblue}{HTML}{7b68ee}  % Load custom colors from colors file
\mytitle{Huanan (Herman) ZHAO}
% Ensure right side margin is not surpassed by bibliography and the right margin is aligned throughout
\RaggedRight


% These \pdfbookmark lines create bookmarks in the exported PDF document that display in the left pane.
% Value in [] sets the indentation level of the bookmark

\pdfbookmark[1]{Address}{}
\section*{Address}
School of Life Sciences, 

Tsinghua University,

30 Shuangqing Road, Haidian District, Beijing 100084, P. R. China

E-mail (work):~\href{mailto:zhn19@mails.tsinghua.edu.cn}{zhn19@mails.tsinghua.edu.cn}

E-mail (home):~\href{mailto:hermanzhaozzzz@gmail.com}{hermanzhaozzzz@gmail.com}

Tel:~(+86)~166~1995~4871

% \pdfbookmark[1]{Personal Information}{}
% \section*{Personal Information}
% Date of birth: September $21^{st}$, 1995 

Gender: Male 

Nationality: P. R. China


\pdfbookmark[1]{Education}{}
\section*{Education}
% Add your educational background here!

% NOTE: If you want to remove the "Expected" footnote, you will want to remove:
% - Directly below: \renewcommand, \setcounter
% - In the table: \footnotemark in the left column
% - After the table: \footnotetext, \renewcommand, \setcounter

% Different numbers in "\setcounter{footnote}{0}" use different symbols
\renewcommand{\thefootnote}{\fnsymbol{footnote}}
\setcounter{footnote}{0}
\begin{longtable}{p{0.6\textwidth} p{0.4\textwidth}}
%     % Use custom symbol footnote for "expected"
    \textit{\textbf{Tsinghua University, Beijing, China}} & \textit{\textbf{Sept. 2019 $\sim$ Present}} \\
    \textbf{Ph.D.} in Bioinformatics \& Gene Editing & Advisor: \href{https://www.bio.pku.edu.cn/enhomes/news/teacher_dis/91.html}{Prof. Chengqi Yi} \\
    & \\
    \textit{\textbf{Peking University, Beijing, China}} & \textit{\textbf{Sept. 2019 $\sim$ Present}} \\
    \textbf{Ph.D. joint training} in Bioinformatics \& Gene Editing & Advisor: \href{https://www.bio.pku.edu.cn/enhomes/news/teacher_dis/91.html}{Prof. Chengqi Yi} \\
    & \\
    \textit{\textbf{Northwest A\&F University, Beijing, China}} & \textit{\textbf{Sept. 2014 $\sim$ June 2019}} \\
    \textbf{BVSc} in Veterinary Medicine & Advisor: \href{https://faculty.nwu.edu.cn/gaomingqing/en/index.htm}{Prof. Mingqing Gao} \\
    & \\
    \textit{\textbf{China Agricultural University}} & \textit{\textbf{Sept. 2016 $\sim$ June 2018}} \\
    \textbf{BVSc exchange student} in Veterinary Medicine & Advisor: \href{https://cvm.cau.edu.cn/art/2017/9/12/art_41957_71.html}{Prof. Jian Gao} \\
\end{longtable}


% Add text for the custom footnote
\footnotetext[1]{Expected.}

% Restore the default footnote numbering
\renewcommand{\thefootnote}{\arabic{footnote}}
\setcounter{footnote}{1}


\pdfbookmark[1]{Research Experience}{exp_research}
\section*{Research Experience}
\label{exp_research}
% List your research experience here

% Add equal contribution dagger
% \vspace{-.75em}
\small
$*$ Bioinformatics research experience \hspace{2em} $\dagger$ Wet-lab research experience

\normalsize
\vspace{10pt}

\textbf{Ph.D. student in \href{https://www.bio.pku.edu.cn/enhomes/news/teacher_dis/91.html}{Prof. Chengqi Yi}'s \href{https://yilab.org.cn/}{lab}, Peking University} \hfill Feb. 2020 $\sim$ Present

{\small
$*$ Independently developed mining methods to identify novel type VI CRISPR systems.\\
$*$ Performed off-target profiling of genome editing tools, including CBE, ABE, and DdCBE\cite{lei2022mitochondrial,rao2023characterizing,lei2021detect,lei2023detect}.\\
$*$ \textbf{Ph.D. thesis:} \textit{``Evaluation, Optimization, and Mechanism Studies of Cytosine Base Editors''.}
}

\vspace{5pt}

\textbf{Rotating student in \href{https://life.tsinghua.edu.cn/lifeen/info/1034/1077.htm}{Prof. Wei Xie}'s \href{http://www.xielab.org.cn/}{lab}, Tsinghua University} \hfill Nov. 2019 $\sim$ Jan. 2020

{\small
$*$ Bioinformatics analysis of ultrasensitive Ribo-seq and RNA-seq.\\
$\dagger$ RNA-seq library building.
}

\vspace{5pt}

\textbf{Rotating student in \href{https://life.tsinghua.edu.cn/lifeen/info/1034/1083.htm}{Prof. Zhi John Lu}'s \href{https://lulab.life.tsinghua.edu.cn/labhome/home/}{lab}, Tsinghua University} \hfill Sept. 2019 $\sim$ Oct. 2019

{\small
$*$ RNA structure prediction\cite{zhu2021integrative}, SHAPE-seq analysis.
}


\vspace{5pt}


\textbf{Intern in \href{https://www.bio.pku.edu.cn/enhomes/news/teacher_dis/43.html}{Prof. Yulong Li}'s \href{http://www.yulonglilab.org/}{lab}, Peking University} \hfill Mar. 2019 $\sim$ Aug. 2019

{\small
$\dagger$ Evolution of artificial point mutations for genetically encoded neuropeptide sensors.
% Cell culture, molecular cloning (Gibson homologous recombination method) and transfections.
}


\vspace{5pt}


\textbf{B.V.Sc Thesis Training in \href{https://faculty.nwu.edu.cn/gaomingqing/zh_CN/index.htm}{Prof. Mingqing Gao}'s lab, Northwest A\&F University} \hfill Sept. 2018 $\sim$ Feb. 2019

{\small
$\dagger$ The crosstalk between the NF$\kappa$B and Nrf2 pathways. Nuclear and cytoplasmic separation was performed, followed by western blot and immunofluorescent staining.\\
$\dagger$ \textbf{B.V.Sc Thesis:} \textit{``The Role and Mechanism of N-3 Polyunsaturated Fatty Acids in the Inflammatory Response of Mammary Gland Epithelial Cells in Dairy Cows''.}
}


\vspace{5pt}


\textbf{Intern in \href{https://cvm.cau.edu.cn/art/2017/9/12/art_41957_71.html}{Prof. Jian Gao}'s lab, China Agricultural University} \hfill Sept. 2016 $\sim$ Aug. 2018

{\small
$\dagger$ Molecular epidemiology and distribution of antimicrobial resistance genes of \textit{Staphylococcus} species\cite{zhao2018detection,qu2019molecular}.
}

\vspace{5pt}

\textbf{Intern in \href{https://dyxy.nwsuaf.edu.cn/en/People/FullProfessor/5e6012bf013a499c946289472b285795.htm}{Prof. Qin Zhao}'s lab. Northwest A\&F University} \hfill Sept. 2015 $\sim$ Jun. 2016

{\small
$\dagger$ Entry-level research training.
}


\pdfbookmark[1]{Publications}{pubs}
\section*{Publications}
\label{pubs}

% Add equal contribution dagger
\vspace{-.75em}
\small
$\dagger$ co-first author
\normalsize


\nocite{*} % Ensures uncited items are included
\printbibliography[
    type=article, % Only include @article ref.bib items
    heading=none, % Do not include header. Gives us more control.
    resetnumbers=true, % Start item counter from zero
    % keyword={A,R,P} % Include items in ref.bib with keyword={J}
]

\pdfbookmark[1]{Presentations}{presentations}
\section*{Presentations}
\label{presentations}
\begin{enumerate}
    \item \textbf{``The development of genome, genome sequencing and sequencers.''}
    \textit{A lecture presented to high school students as part of a science popularization program at the State Key Laboratory of Protein and Plant Gene Research, Peking University, Beijing, China, in 2024.}
    \textbf{(Oral Lecture)}
    \item \textbf{``Safety evaluation for genome editing.''}
    \textit{A report delivered at the Beijing Institute of Life Science, Chinese Academy of Sciences, Beijing, China, in 2023.} 
    \textbf{(Invited Oral Report)}.
    \item \textbf{``Mitochondrial base editor induces substantial nuclear off-target mutations.''}
    \textit{A presentation given at the annual meeting of the School of Life Sciences, Tsinghua University, Xiong'an New Area, Hebei Province, China, in 2023.} 
    \textbf{Oral Presentation}.
    \item \textbf{``Mitochondrial base editor induces substantial nuclear off-target mutations.''}
    \textit{A report presented at the Center for Life Sciences ``1$+$X'' Forum, Beijing, China, in 2022}. 
    \textbf{(Invited Oral Report)}.
    \item \textbf{``Mitochondrial base editor induces substantial nuclear off-target mutations.''}
    \textit{A presentation given at an academic exchange conference for doctoral students in the School of Life Sciences, Tsinghua University, Beijing, China, in 2022.} 
    \textbf{Oral Presentation}.
    \item \textbf{``Sharing My Doctoral and Undergraduate University Life and Research Experiences.''}
    \textit{An invited online presentation to undergraduate students at the School of Medicine, Northwest University, Xi’an, Shaanxi Province, China, in 2022.} 
    \textbf{Invited Virtual Meeting}.
\end{enumerate}


\pdfbookmark[1]{Awards \& Honors}{awards}
\section*{Awards \& Honors}
\label{awards}
\textbf{First Class scholarship for comprehensive excellence} \hfill Oct. 2023 \\
{\small Tsinghua University\hfill  Beijing, China}

\vspace{4pt}

\textbf{Excellent Team for Big Data Practice Course Project} \hfill Nov. 2022 \\
{\small Awarded for research: ``Bioinformatics Research on Childhood Hearing Loss'', Tsinghua University \hfill Beijing, China}

\vspace{4pt}

\textbf{Outstanding Oral Report Award} \hfill Sept. 2022 \\
{\small Presented at the 8$^{th}$ Academic Forum on Molecular Biochemistry, Peking University \hfill Beijing, China}

\vspace{4pt}

% \textbf{Second Class scholarship for comprehensive excellence} \hfill Nov. 2021 \\
% {\small Tsinghua University \hfill Beijing, China}

\vspace{4pt}

\textbf{Star of Graduate Students} \hfill Dec. 2020 \\
{\small Top student from each college, Tsinghua University \hfill Beijing, China}

\vspace{4pt}

\textbf{``Hundred'' Outstanding Undergraduate Theses} \hfill Jun. 2019 \\
{\small Top student from each college, Northwest A\&F University \hfill Shaanxi, China}

\vspace{4pt}

\textbf{Outstanding Undergraduate Thesis at the University Level} \hfill May. 2019 \\
{\small Northwest A\&F University \hfill Shaanxi, China}

\vspace{4pt}

\textbf{Top Ten Excellent Graduate in Scientific Research} \hfill May. 2019 \\
{\small College of Veterinary Medicine, Northwest A\&F University \hfill Shaanxi, China}

\vspace{4pt}

% \textbf{Boehringer-Ingelheim Scholarship} \hfill Oct. 2019 \\
% {\small Boehringer-Ingelheim \hfill Shaanxi, China}
    
% \vspace{4pt}

% \textbf{Silver Prize for National-Level Undergraduate Research Training Program} \hfill 2018 \\
% {\small China Agricultural University \hfill Beijing, China}

% \vspace{4pt}

% \textbf{Golden Prize for University-Level Undergraduate Entrepreneurship Training Program} \hfill Jun. 2017 \\
% {\small Northwest A\&F University \hfill Shaanxi, China}

% \vspace{4pt}

% \textbf{Silver Prize for the ``Challenge Cup'' Entrepreneurship Plan Competition} \hfill 2016 \\
% {\small Northwest A\&F University \hfill Shaanxi, China}

\vspace{4pt}

\textbf{Second Prize for Chemical Experiment Skills Competition} \hfill 2016 \\
{\small Northwest A\&F University \hfill Shaanxi, China}

% \vspace{4pt}

% \textbf{Silver Prize for the ``Internet+'' Innovation and Entrepreneurship Competition (Shaanxi)} \hfill Sept. 2016 \\
% {\small Win honors for Northwest A\&F University, ``Internet+'' Shaanxi Organizing Committee \hfill Shaanxi, China}

\vspace{4pt}

\textbf{National Scholarship} \hfill 2015 $\sim$ 2016 \\
{\small The Highest Level of Award for Undergraduate. \hfill Ministry of Education of P. R. China}\\


\vspace{4pt}

\textbf{First-Class Scholarship for Professional Excellence} \hfill 2015 $\sim$ 2019 \\
{\small 2015,2016,2017,2018,2019, five-time winner, Northwest A\&F University \hfill Shaanxi, China}

% \vspace{4pt}

% \textbf{Merit Student} \hfill 2015 $\sim$ 2019 \\
% {\small 2015,2016,2017,2018,2019, five-time winner, Northwest A\&F University \hfill Shaanxi, China}



\pdfbookmark[1]{Tools \& Software}{tools}
\section*{Tools \& Software}
\label{tools}
\subsection*{\textbf{Selected Open Source Softwares that I developed}}
\begin{itemize}
    \item \href{https://github.com/hermanzhaozzzz/.my_shell_envs}{.my\_shell\_envs}: A fast deployment tool for shell (zsh) envs.
    \item \href{https://github.com/hermanzhaozzzz/bioat}{bioat}: A python package \& a command line toolkit for Bioinformatics and data science. (\href{https://pypi.org/project/bioat/}{PyPi} | \href{https://github.com/hermanzhaozzzz/bioat}{GitHub}). Demonstrated at (\href{https://github.com/hermanzhaozzzz/bioat}{README.md}).
    \item \href{https://github.com/hermanzhaozzzz/My-Opencore-EFI-for-AMD3900X-5700XT-TUF-x570-Hackintosh}{My-Opencore-EFI-for-AMD3900X-5700XT-TUF-x570-Hackintosh}: My Opencore EFI for AMD3900X-5700XT-TUF-x570 Hackintosh.
\end{itemize}

\subsection*{\textbf{Selected Open Source Bioinformatic Analysis Tools that I developed}}
\begin{itemize}
    \item \href{https://github.com/hermanzhaozzzz/snakepipes_RNA-seq}{snakepipes\_RNA-seq}: A ``Snakemake'' pipeline for RNA-seq data analysis.
    \item \href{https://github.com/hermanzhaozzzz/snakepipes_Hi-C}{snakepipes\_Hi-C}: A ``Snakemake'' pipeline for Hi-C data analysis.
    \item \href{https://github.com/hermanzhaozzzz/snakepipes_ATAC-seq}{snakepipes\_ATAC-seq}: A ``Snakemake'' pipeline for ATAC-seq data analysis.
    \item \href{https://github.com/hermanzhaozzzz/snakepipes_ChIP-seq}{snakepipes\_ChIP-seq}: A ``Snakemake'' pipeline for ChIP-seq data analysis.
    \item \href{https://github.com/hermanzhaozzzz/snakepipes_detect-seq}{snakepipes\_detect-seq}: A ``Snakemake'' pipeline for detect-seq data analysis.
    \item \href{https://github.com/hermanzhaozzzz/snakepipes_fastqc-multiqc}{snakepipes\_fastqc-multiqc}: A ``Snakemake'' pipeline for quality control of NGS/HTS data.
\end{itemize}

\subsection*{\textbf{Selected Open Source Softwares/Tools that I contribute codes or issues}}
\begin{itemize}
    \item \href{https://github.com/owkin/PyDESeq2}{PyDESeq2}: PyDESeq2 is a python implementation of the DESeq2 method for differential 
    expression analysis (DEA) with bulk RNA-seq data, originally in R. It aims to facilitate DEA experiments for python users.
    \item \href{https://github.com/DessimozLab/fold_tree}{fold\_tree}: This repo contains the scripts and snakemake workflow for making and benchmarking structure based trees using Foldseek.
    \item \href{https://github.com/python-poetry/poetry}{poetry}: Poetry is a tool for dependency management and packaging in Python.
\end{itemize}




\pdfbookmark[1]{Selected Tutorials}{tutorials}
\section*{Selected Tutorials}
\label{tutorials}
\begin{enumerate}
    \item Quick deployment of a user-friendly command line interface on MacOS / Linux / Windows,
    \href{https://zhuanlan.zhihu.com/p/648520368}{Zhihu},
    \href{https://github.com/hermanzhaozzzz/.my_shell_envs}{GitHub}.
    \item Practical application of Hi-C | Reiterating the main conclusions of a Hi-C paper,
    \href{https://zhuanlan.zhihu.com/p/542713896}{Zhihu} [\href{https://zhuanlan.zhihu.com/p/542713896}{part1}, 
    \href{https://zhuanlan.zhihu.com/p/543987644}{part2},
    \href{https://zhuanlan.zhihu.com/p/545657147}{part3}].
    \item Mapping raw reads of base substitution sequencing technology using Hisat-3n: Efficiency and performance testing,
    \href{https://zhuanlan.zhihu.com/p/386371449}{Zhihu}.
\end{enumerate}


\pdfbookmark[1]{Teaching}{teaching}
\section*{Teaching}
\label{teaching}
\textbf{\href{https://www.bioinfo.info/p/t_pc/goods_pc_detail/goods_detail/course_2SvfNlIVzrKfOcexHk9Nute5Bhd}{Bioinformatics analysis course using Python (Online)}}

This 30-hour course, led by myself (principal instructor) and Dr. Haowei Meng (assistant instructor), 
combines foundational programming skills with real-world project experience in bioinformatics analysis. 
Offered on two platforms, the course has received over 500 enrollments and 20+ reviews, 
all of which are 5-star ratings (out of 5).


\pdfbookmark[1]{Expertise}{expertise}
\section*{Expertise}
\label{expertise}
\textbf{High-throughput Sequencing Analysis}
% \vspace{-1em}
% \begin{multicols}{3} % 开始两列环境
% \begin{itemize}
%     \item RNA-seq
%     \item Hi-C
%     \item ATAC-seq
%     \item DNase-seq
% \end{itemize}  
  
% \columnbreak % 强制换到第二列(如果需要)  
  
% \begin{itemize}  
%     \item MNase-seq
%     \item ChIP-seq
%     \item Ribo-seq
%     \item SHAPE-seq
% \end{itemize}  
  
% \columnbreak % 强制换到第二列(如果需要)  
  
% \begin{itemize}  
%     \item scRNA-seq
%     \item Detect-seq
%     \item Targeted deep sequencing
%     \item Metagenome analysis
% \end{itemize}
% \end{multicols}
% \vspace{-1em}

\textbf{CRISPR-Cas system mining}\\
% Piler-CR, Prodigal, CRISPRCasFinder, \href{https://github.com/hermanzhaozzzz/bioat}{BioAT} (I developed).


\textbf{Sequence Analysis}
% \begin{itemize}
%     \item Alignment tools: BWA, Bowtie, STAR, Hisat2/3n, Bismark, Biopython (package), Pysam (package).
%     \item Sequence comparison/search tools: BLAST kits, MMSeqs2, FastTree, ClustalW.
%     \item Region comparison/search tools: BEDtools, deepTools.
%     \item Variant calling tools: GATK, VarScan2.
% \end{itemize}

\textbf{Structure Analysis}
\begin{itemize}
    \item 3D genome interactions: GENOVA, Hi-C Pro, Hi-C Explorer.
    \item Structure prediction tools: AlphaFold2/3, Colabfold, ESMFold.
    \item Structure comparison/search tools: Foldseek, Dali.
\end{itemize}

\textbf{Function Analysis}
\begin{itemize}
    \item Enrichment analysis tools: clusterProfiler (R), DESeq2(R), David, KEGG, GSEA.
    \item Functional prediction tools: InterProScan, Pfam, Rfam, HMMER, HOMER, MEME Suite.
\end{itemize}


\textbf{Core Development Skills}\\

Seasoned \underline{Python} \href{https://github.com/hermanzhaozzzz}{developer} 
and \href{https://www.bioinfo.info/p/t_pc/goods_pc_detail/goods_detail/course_2SvfNlIVzrKfOcexHk9Nute5Bhd}{instructor}.\\
Proficient in data analysis and visualization via \underline{R} in a ``tidyverse + ggplot2'' manner.
% \vspace{-1em}
% \begin{multicols}{3} % 开始两列环境
% \begin{itemize}  
%     \item \textit{Web Development:}  
%     \begin{itemize}
%         \item Django  
%         \item Flask  
%     \end{itemize}  
  
%     \item \textit{Command-Line Tools:}  
%     \begin{itemize}  
%         \item click  
%         \item argparse  
%         \item Fire  
%     \end{itemize}  
  
%     \item \textit{Crawler Development:}  
%     \begin{itemize}  
%         \item Requests  
%         \item BeautifulSoup  
%         \item Selenium  
%         \item Playwright  
%     \end{itemize}  
  
%     \item \textit{Data Analysis:}  
%     \begin{itemize}  
%         \item Pandas  
%         \item NumPy  
%         \item Polars  
%         \item tidyverse (R)  
%     \end{itemize}  
% \end{itemize}  

% 由于multicols环境不会自动平衡列高,我们可能需要手动调整内容  
% 以下内容可以移至第二列或根据需要进行调整  
% \columnbreak % 强制换到第二列(如果需要)  
  
% \begin{itemize}  
%     \item \textit{Visualization:}  
%     \begin{itemize}  
%         \item Matplotlib  
%         \item Seaborn  
%         \item lets-plot  
%         \item ggplot2 (R)  
%     \end{itemize}  

      
%     \item \textit{Machine Learning Frameworks:}  
%     \begin{itemize}  
%         \item scikit-learn  
%         \item PyTorch  
%     \end{itemize}  



%     \item \textit{Big Data Capability:}  
%     \begin{itemize}  
%         \item MySQL (pymysql)  
%         \item Dask  
%         \item pySpark  
%     \end{itemize}  
  
%     \item \textit{Shell Scripting:}  
%     \begin{itemize}  
%         \item Bash  
%         \item Zsh  
%         \item PowerShell  
%     \end{itemize}  
  
% \end{itemize}  
  
% 由于multicols环境不会自动平衡列高,我们可能需要手动调整内容  
% 以下内容可以移至第二列或根据需要进行调整  
% \columnbreak % 强制换到第二列(如果需要)  
  
% \begin{itemize}  
%     \item \textit{Platforms:}  
%     \begin{itemize}  
%         \item Linux (remote)
%         \item macOS (local)
%         \item Windows (second choice)
%     \end{itemize} 

%     \item \textit{Task Management Systems/Tools:}  
%     \begin{itemize}  
%         \item Slurm
%         \item Torque
%         \item Snakemake
%     \end{itemize} 

%     \item \textit{Environment/Version Control:}  
%     \begin{itemize}  
%         \item Git
%         \item Docker
%         \item conda
%         \item poetry
%     \end{itemize} 

%     \item \textit{Typesetting tools:}  
%     \begin{itemize}  
%         \item \LaTeX
%         \item Beamer
%         \item Markdown
%     \end{itemize} 
% \end{itemize}  
  
% \end{multicols} % 结束两列环境  
% \vspace{-1em}

\textbf{Animal Experiment Skills}

% \begin{itemize}
%     \item Anatomical experience with fish, frog, fowl, mice, rabbit, pig, and dog.
%     \item Blood sampling and injection for rabbit, rat, mice, cat, dog, cow, goat, and horse.
% \end{itemize}


\textbf{Wet Experiment Skills}
\vspace{-1em}
\begin{multicols}{3}
\begin{itemize}
    \item \textit{Molecular cloning}:
    % \begin{itemize}
    %     \item Primer design
    %     \item DNA/RNA extraction
    %     \item PCR
    %     \item Gel electrophoresis
    % \end{itemize}
\end{itemize}  
  
% 由于multicols环境不会自动平衡列高,我们可能需要手动调整内容  
% 以下内容可以移至第二列或根据需要进行调整  
\columnbreak % 强制换到第二列(如果需要)  

\begin{minipage}[t]{0.35\textwidth} % 第二列宽度为35%  
    \begin{itemize} 
        \item \textit{Mammalian cell experiments}:
        % \begin{itemize}
        %     \item Cell culture
        %     \item Transfection
        %     \item Western blot
        %     \item Immunofluorescence staining
        % \end{itemize}
    \end{itemize}
\end{minipage}  


% 由于multicols环境不会自动平衡列高,我们可能需要手动调整内容  
% 以下内容可以移至第二列或根据需要进行调整  
\columnbreak % 强制换到第二列(如果需要)  
  
\begin{itemize} 
    \item \textit{Bacteria experiments}:
    % \begin{itemize}
    %     \item Bacteria culture
    %     \item Transformation
    %     \item Plasmid amplification
    % \end{itemize}
\end{itemize}  
\end{multicols}
\vspace{-1em}

% Pretty ending with the date last updated
\centering
\rule{0.25\linewidth}{0.4pt}\\
\medskip
Last updated: \today\\

\end{document}

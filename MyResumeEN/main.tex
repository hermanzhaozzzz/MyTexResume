% ========================================================
% This document is a customizable CV/Resume template built using LaTeX.
% The template is designed for easy customization and clear structure.
%
% Author: Matthew DeVerna (www.matthewdeverna.com)
% Date: 2024
% Design: Cased on Hause Lin's CV (hauselin.com)
% 
% Project Overview:
% -----------------
% This LaTeX document is designed to help you create a professional CV or 
% resume with ease. It uses as little fancy LaTex functionality or custom functions as possible to maximize its longterm durability and flexibility.
% The document is structured into multiple sections, each loaded from 
% separate subfiles for modularity and ease of maintenance. 
%
% Key Features:
% -------------
% - Customizable sections: Education, Research Experience, Awards, Publications, etc.
% - Bookmarks in the PDF for easy navigation
% - Styled bibliography with BibLaTeX
% - Hyperlinked email and website
% - FontAwesome icons for additional styling
%
% Getting Started:
% ----------------
% 1. Customize your personal information by modifying the \mytitle command.
% 2. Add your content to the respective subfiles (e.g., education.tex, exp_research.tex).
% 3. Update the bibliography file (ref.bib) with your publications and categorize them with keywords.
%
% Important Notes:
% ----------------
% - This main file includes the overall structure and settings. 
% - Each section has its own detailed instructions for further customization.
%
% ========================================================

\documentclass[11pt]{article} % Choose the document class and font size
\usepackage[margin=1in]{geometry} % Page layout settings

% Set the citation style
\usepackage[
    backend=biber,      % Specifies the backend to be used by BibLaTeX for processing the bibliography. 'biber' is the default backend.
    maxnames=20,        % Limits the maximum number of author names to display before abbreviating with "et al."
    style=nature,       % Sets the citation style to 'nature,' which is commonly used in scientific papers.
    sorting=ydnt,       % Specifies the sorting order of entries in the bibliography:
                        % y - year (descending)
                        % d - descending order
                        % n - name
                        % t - title
    defernumbers=true,  % Delays the assignment of citation numbers until the end of the document, allowing for the correct order of citations within each bibliography section.
]{biblatex}
\addbibresource{ref.bib} % Adds the bibliography resource file 'ref.bib' containing all the references.

% Allows columns that stretch across pages
\usepackage{longtable}

% Table functionality and beautification (not strictly needed)
\usepackage{bookmark}

% Use icons, if you want.
% All available icons: http://mirrors.ibiblio.org/CTAN/fonts/fontawesome5/doc/fontawesome5.pdf
\usepackage{fontawesome}

% Allows font justification control (needed for clean pub-list formatting)
\usepackage{ragged2e}

% For underlining with line breaks
\usepackage{soul} 

% All fonts: https://tug.org/FontCatalogue/
\usepackage{kpfonts} % More professional font
% \usepackage[default]{sourcecodepro} % Code-like font
\usepackage[T1]{fontenc}

% Control hyperlinks and colors
% CUSTOM COLORS INCLUDED DIRECTLY AFTER \begin{document}
\usepackage{xcolor}
\usepackage{hyperref}
\hypersetup{
    colorlinks=true,        % Enable colored links
    breaklinks=true,        % Allow links to break across lines
    linkcolor=cornflowerblue,    % Color of internal links
    urlcolor=cornflowerblue,     % Color of URL links
    anchorcolor=cornflowerblue,  % Color of anchors
    citecolor=cornflowerblue,    % Color of citations
    pdftitle={Your Title},    % Title of the PDF
    pdfauthor={Your Name}, % Author of the PDF
    bookmarksopen=true,      % Open bookmarks panel at start
}

%%% CONVENIENCE FUNCTIONS GO HERE %%%
%%% ----------------------------- %%%
\newcommand{\mytitle}[4]{
  \begin{center}
    \Large\textbf{#1}\normalsize \\ % Name in large bold font
    \href{mailto:#2}{#2} \\ % Email with mailto: link
    \href{https://#3}{#3} \\ % Website with link
    #4 % Address
  \end{center}
}
%%% ----------------------------- %%%


\begin{document}
% Set custom colors here (imported directly after \begin{document})
% The below use HTML hex codes.
% More HTML hex codes: https://encycolorpedia.com/html
\definecolor{firebrick}{HTML}{b22222} 
\definecolor{darkslategrey}{HTML}{2f4f4f} 
\definecolor{cornflowerblue}{HTML}{6495ed} 
\definecolor{mediumslateblue}{HTML}{7b68ee}  % Load custom colors from colors file
\mytitle{Your Name}{person@gmail.com}{yourwebsite.com}{Your Address\\Goes here\\This field can be excluded} % Insert your custom title


% Ensure right side margin is not surpassed by bibliography and the right margin is aligned throughout
\RaggedRight


% These \pdfbookmark lines create bookmarks in the exported PDF document that display in the left pane.
% Value in [] sets the indentation level of the bookmark
\pdfbookmark[1]{Education}{}
\section*{Education}
% Add your educational background here!

% NOTE: If you want to remove the "Expected" footnote, you will want to remove:
% - Directly below: \renewcommand, \setcounter
% - In the table: \footnotemark in the left column
% - After the table: \footnotetext, \renewcommand, \setcounter

% Different numbers in "\setcounter{footnote}{0}" use different symbols
\renewcommand{\thefootnote}{\fnsymbol{footnote}}
\setcounter{footnote}{0}
\begin{longtable}{p{0.6\textwidth} p{0.4\textwidth}}
%     % Use custom symbol footnote for "expected"
    \textit{\textbf{Tsinghua University, Beijing, China}} & \textit{\textbf{Sept. 2019 $\sim$ Present}} \\
    \textbf{Ph.D.} in Bioinformatics \& Gene Editing & Advisor: \href{https://www.bio.pku.edu.cn/enhomes/news/teacher_dis/91.html}{Prof. Chengqi Yi} \\
    & \\
    \textit{\textbf{Peking University, Beijing, China}} & \textit{\textbf{Sept. 2019 $\sim$ Present}} \\
    \textbf{Ph.D. joint training} in Bioinformatics \& Gene Editing & Advisor: \href{https://www.bio.pku.edu.cn/enhomes/news/teacher_dis/91.html}{Prof. Chengqi Yi} \\
    & \\
    \textit{\textbf{Northwest A\&F University, Beijing, China}} & \textit{\textbf{Sept. 2014 $\sim$ June 2019}} \\
    \textbf{BVSc} in Veterinary Medicine & Advisor: \href{https://faculty.nwu.edu.cn/gaomingqing/en/index.htm}{Prof. Mingqing Gao} \\
    & \\
    \textit{\textbf{China Agricultural University}} & \textit{\textbf{Sept. 2016 $\sim$ June 2018}} \\
    \textbf{BVSc exchange student} in Veterinary Medicine & Advisor: \href{https://cvm.cau.edu.cn/art/2017/9/12/art_41957_71.html}{Prof. Jian Gao} \\
\end{longtable}


% Add text for the custom footnote
\footnotetext[1]{Expected.}

% Restore the default footnote numbering
\renewcommand{\thefootnote}{\arabic{footnote}}
\setcounter{footnote}{1}


\pdfbookmark[1]{Research Experience}{exp_research}
\section*{Research Experience}
\label{exp_research}
% List your research experience here

% Add equal contribution dagger
% \vspace{-.75em}
\small
$*$ Bioinformatics research experience \hspace{2em} $\dagger$ Wet-lab research experience

\normalsize
\vspace{10pt}

\textbf{Ph.D. student in \href{https://www.bio.pku.edu.cn/enhomes/news/teacher_dis/91.html}{Prof. Chengqi Yi}'s \href{https://yilab.org.cn/}{lab}, Peking University} \hfill Feb. 2020 $\sim$ Present

{\small
$*$ Independently developed mining methods to identify novel type VI CRISPR systems.\\
$*$ Performed off-target profiling of genome editing tools, including CBE, ABE, and DdCBE\cite{lei2022mitochondrial,rao2023characterizing,lei2021detect,lei2023detect}.\\
$*$ \textbf{Ph.D. thesis:} \textit{``Evaluation, Optimization, and Mechanism Studies of Cytosine Base Editors''.}
}

\vspace{5pt}

\textbf{Rotating student in \href{https://life.tsinghua.edu.cn/lifeen/info/1034/1077.htm}{Prof. Wei Xie}'s \href{http://www.xielab.org.cn/}{lab}, Tsinghua University} \hfill Nov. 2019 $\sim$ Jan. 2020

{\small
$*$ Bioinformatics analysis of ultrasensitive Ribo-seq and RNA-seq.\\
$\dagger$ RNA-seq library building.
}

\vspace{5pt}

\textbf{Rotating student in \href{https://life.tsinghua.edu.cn/lifeen/info/1034/1083.htm}{Prof. Zhi John Lu}'s \href{https://lulab.life.tsinghua.edu.cn/labhome/home/}{lab}, Tsinghua University} \hfill Sept. 2019 $\sim$ Oct. 2019

{\small
$*$ RNA structure prediction\cite{zhu2021integrative}, SHAPE-seq analysis.
}


\vspace{5pt}


\textbf{Intern in \href{https://www.bio.pku.edu.cn/enhomes/news/teacher_dis/43.html}{Prof. Yulong Li}'s \href{http://www.yulonglilab.org/}{lab}, Peking University} \hfill Mar. 2019 $\sim$ Aug. 2019

{\small
$\dagger$ Evolution of artificial point mutations for genetically encoded neuropeptide sensors.
% Cell culture, molecular cloning (Gibson homologous recombination method) and transfections.
}


\vspace{5pt}


\textbf{B.V.Sc Thesis Training in \href{https://faculty.nwu.edu.cn/gaomingqing/zh_CN/index.htm}{Prof. Mingqing Gao}'s lab, Northwest A\&F University} \hfill Sept. 2018 $\sim$ Feb. 2019

{\small
$\dagger$ The crosstalk between the NF$\kappa$B and Nrf2 pathways. Nuclear and cytoplasmic separation was performed, followed by western blot and immunofluorescent staining.\\
$\dagger$ \textbf{B.V.Sc Thesis:} \textit{``The Role and Mechanism of N-3 Polyunsaturated Fatty Acids in the Inflammatory Response of Mammary Gland Epithelial Cells in Dairy Cows''.}
}


\vspace{5pt}


\textbf{Intern in \href{https://cvm.cau.edu.cn/art/2017/9/12/art_41957_71.html}{Prof. Jian Gao}'s lab, China Agricultural University} \hfill Sept. 2016 $\sim$ Aug. 2018

{\small
$\dagger$ Molecular epidemiology and distribution of antimicrobial resistance genes of \textit{Staphylococcus} species\cite{zhao2018detection,qu2019molecular}.
}

\vspace{5pt}

\textbf{Intern in \href{https://dyxy.nwsuaf.edu.cn/en/People/FullProfessor/5e6012bf013a499c946289472b285795.htm}{Prof. Qin Zhao}'s lab. Northwest A\&F University} \hfill Sept. 2015 $\sim$ Jun. 2016

{\small
$\dagger$ Entry-level research training.
}

\pdfbookmark[1]{Awards \& Honors}{awards}
\section*{Awards \& Honors}
\label{awards}
\textbf{First Class scholarship for comprehensive excellence} \hfill Oct. 2023 \\
{\small Tsinghua University\hfill  Beijing, China}

\vspace{4pt}

\textbf{Excellent Team for Big Data Practice Course Project} \hfill Nov. 2022 \\
{\small Awarded for research: ``Bioinformatics Research on Childhood Hearing Loss'', Tsinghua University \hfill Beijing, China}

\vspace{4pt}

\textbf{Outstanding Oral Report Award} \hfill Sept. 2022 \\
{\small Presented at the 8$^{th}$ Academic Forum on Molecular Biochemistry, Peking University \hfill Beijing, China}

\vspace{4pt}

% \textbf{Second Class scholarship for comprehensive excellence} \hfill Nov. 2021 \\
% {\small Tsinghua University \hfill Beijing, China}

\vspace{4pt}

\textbf{Star of Graduate Students} \hfill Dec. 2020 \\
{\small Top student from each college, Tsinghua University \hfill Beijing, China}

\vspace{4pt}

\textbf{``Hundred'' Outstanding Undergraduate Theses} \hfill Jun. 2019 \\
{\small Top student from each college, Northwest A\&F University \hfill Shaanxi, China}

\vspace{4pt}

\textbf{Outstanding Undergraduate Thesis at the University Level} \hfill May. 2019 \\
{\small Northwest A\&F University \hfill Shaanxi, China}

\vspace{4pt}

\textbf{Top Ten Excellent Graduate in Scientific Research} \hfill May. 2019 \\
{\small College of Veterinary Medicine, Northwest A\&F University \hfill Shaanxi, China}

\vspace{4pt}

% \textbf{Boehringer-Ingelheim Scholarship} \hfill Oct. 2019 \\
% {\small Boehringer-Ingelheim \hfill Shaanxi, China}
    
% \vspace{4pt}

% \textbf{Silver Prize for National-Level Undergraduate Research Training Program} \hfill 2018 \\
% {\small China Agricultural University \hfill Beijing, China}

% \vspace{4pt}

% \textbf{Golden Prize for University-Level Undergraduate Entrepreneurship Training Program} \hfill Jun. 2017 \\
% {\small Northwest A\&F University \hfill Shaanxi, China}

% \vspace{4pt}

% \textbf{Silver Prize for the ``Challenge Cup'' Entrepreneurship Plan Competition} \hfill 2016 \\
% {\small Northwest A\&F University \hfill Shaanxi, China}

\vspace{4pt}

\textbf{Second Prize for Chemical Experiment Skills Competition} \hfill 2016 \\
{\small Northwest A\&F University \hfill Shaanxi, China}

% \vspace{4pt}

% \textbf{Silver Prize for the ``Internet+'' Innovation and Entrepreneurship Competition (Shaanxi)} \hfill Sept. 2016 \\
% {\small Win honors for Northwest A\&F University, ``Internet+'' Shaanxi Organizing Committee \hfill Shaanxi, China}

\vspace{4pt}

\textbf{National Scholarship} \hfill 2015 $\sim$ 2016 \\
{\small The Highest Level of Award for Undergraduate. \hfill Ministry of Education of P. R. China}\\


\vspace{4pt}

\textbf{First-Class Scholarship for Professional Excellence} \hfill 2015 $\sim$ 2019 \\
{\small 2015,2016,2017,2018,2019, five-time winner, Northwest A\&F University \hfill Shaanxi, China}

% \vspace{4pt}

% \textbf{Merit Student} \hfill 2015 $\sim$ 2019 \\
% {\small 2015,2016,2017,2018,2019, five-time winner, Northwest A\&F University \hfill Shaanxi, China}



\pdfbookmark[1]{Publications}{pubs}
\section*{Publications}
\label{pubs}

% Add equal contribution dagger
\vspace{-.75em}
\small
\faGoogle~\href{https://scholar.google.com/}{Google Scholar}\\
$\dagger \rightarrow$ Equal contribution
\normalsize


\pdfbookmark[2]{Journal Articles}{journal-article}
\subsection*{Journal Articles}
\label{journal-article}
\newrefcontext[labelprefix=J] % Will prefix bibliography numbers with this letter
% Ensures publications which are not cited in the document are included in the above sections
\nocite{*} % Ensures uncited items are included
\printbibliography[
    type=article, % Only include @article ref.bib items
    heading=none, % Do not include header. Gives us more control.
    resetnumbers=true, % Start item counter from zero
    keyword=J % Include items in ref.bib with keyword={J}
]

\pdfbookmark[2]{Peer-reviewed Conference Proceedings}{conferences}
\subsection*{Peer-reviewed Conference Proceedings}
\label{conferences}
\newrefcontext[labelprefix=C]
\printbibliography[type=inproceedings,heading=none,resetnumbers=true,keyword=C]

\pdfbookmark[2]{Working papers}{working-papers}
\subsection*{Working papers}
\label{working-papers}
\newrefcontext[labelprefix=W]
\printbibliography[type=misc,heading=none,resetnumbers=true,keyword=R]
    

\pdfbookmark[1]{Tools \& Software}{tools}
\section*{Tools \& Software}
\label{tools}
\subsection*{\textbf{Selected Open Source Softwares that I developed}}
\begin{itemize}
    \item \href{https://github.com/hermanzhaozzzz/.my_shell_envs}{.my\_shell\_envs}: A fast deployment tool for shell (zsh) envs.
    \item \href{https://github.com/hermanzhaozzzz/bioat}{bioat}: A python package \& a command line toolkit for Bioinformatics and data science. (\href{https://pypi.org/project/bioat/}{PyPi} | \href{https://github.com/hermanzhaozzzz/bioat}{GitHub}). Demonstrated at (\href{https://github.com/hermanzhaozzzz/bioat}{README.md}).
    \item \href{https://github.com/hermanzhaozzzz/My-Opencore-EFI-for-AMD3900X-5700XT-TUF-x570-Hackintosh}{My-Opencore-EFI-for-AMD3900X-5700XT-TUF-x570-Hackintosh}: My Opencore EFI for AMD3900X-5700XT-TUF-x570 Hackintosh.
\end{itemize}

\subsection*{\textbf{Selected Open Source Bioinformatic Analysis Tools that I developed}}
\begin{itemize}
    \item \href{https://github.com/hermanzhaozzzz/snakepipes_RNA-seq}{snakepipes\_RNA-seq}: A ``Snakemake'' pipeline for RNA-seq data analysis.
    \item \href{https://github.com/hermanzhaozzzz/snakepipes_Hi-C}{snakepipes\_Hi-C}: A ``Snakemake'' pipeline for Hi-C data analysis.
    \item \href{https://github.com/hermanzhaozzzz/snakepipes_ATAC-seq}{snakepipes\_ATAC-seq}: A ``Snakemake'' pipeline for ATAC-seq data analysis.
    \item \href{https://github.com/hermanzhaozzzz/snakepipes_ChIP-seq}{snakepipes\_ChIP-seq}: A ``Snakemake'' pipeline for ChIP-seq data analysis.
    \item \href{https://github.com/hermanzhaozzzz/snakepipes_detect-seq}{snakepipes\_detect-seq}: A ``Snakemake'' pipeline for detect-seq data analysis.
    \item \href{https://github.com/hermanzhaozzzz/snakepipes_fastqc-multiqc}{snakepipes\_fastqc-multiqc}: A ``Snakemake'' pipeline for quality control of NGS/HTS data.
\end{itemize}

\subsection*{\textbf{Selected Open Source Softwares/Tools that I contribute codes or issues}}
\begin{itemize}
    \item \href{https://github.com/owkin/PyDESeq2}{PyDESeq2}: PyDESeq2 is a python implementation of the DESeq2 method for differential 
    expression analysis (DEA) with bulk RNA-seq data, originally in R. It aims to facilitate DEA experiments for python users.
    \item \href{https://github.com/DessimozLab/fold_tree}{fold\_tree}: This repo contains the scripts and snakemake workflow for making and benchmarking structure based trees using Foldseek.
    \item \href{https://github.com/python-poetry/poetry}{poetry}: Poetry is a tool for dependency management and packaging in Python.
\end{itemize}



\pdfbookmark[1]{Presentations}{presentations}
\section*{Presentations}
\label{presentations}

% Include any additional details here
% \vspace{-.75em}
% \small
% $\dagger \rightarrow$ Equal contribution
% \normalsize

\pdfbookmark[2]{Talks}{talks}
\subsection*{Talks}
\label{talks}
\newrefcontext[labelprefix=T]
\printbibliography[type=misc,heading=none,resetnumbers=true,keyword=T]

\pdfbookmark[2]{Posters}{posters}
\subsection*{Posters}
\label{posters}
\newrefcontext[labelprefix=P]        
\printbibliography[type=misc,heading=none,resetnumbers=true,keyword=P]

\pdfbookmark[2]{Demonstrations \& Tutorials}{demos}
\subsection*{Demonstrations \& Tutorials}
\label{demos}
\newrefcontext[labelprefix=D] \printbibliography[type=misc,heading=none,resetnumbers=true,keyword=D]


\pdfbookmark[1]{Selected Media Coverage}{media}
\section*{Selected Media Coverage}
\label{media}
%  Include media coverage here
% 
% \begin{longtable}[l]{@{}p{.125\textwidth} p{0.875\textwidth}}

%     2024 & The New York Times, \href{https://nyt.com}{Scientists figure out the meaning of life.}~\cite{art1} (Note how the regular citation functions work throughout the document!) \\
    
%     2024 & AP News, \href{https://apnews.com/}{What even is science?}~\cite{art2} (Point to any project in your ref.bib file with the standard `cite' function) \\

    
% \end{longtable}

 % Input lines load the material from the subdocuments


\pdfbookmark[1]{Teaching}{teaching}
\section*{Teaching}
\label{teaching}
\textbf{\href{https://www.bioinfo.info/p/t_pc/goods_pc_detail/goods_detail/course_2SvfNlIVzrKfOcexHk9Nute5Bhd}{Bioinformatics analysis course using Python (Online)}}

This 30-hour course, led by myself (principal instructor) and Dr. Haowei Meng (assistant instructor), 
combines foundational programming skills with real-world project experience in bioinformatics analysis. 
Offered on two platforms, the course has received over 500 enrollments and 20+ reviews, 
all of which are 5-star ratings (out of 5).


\pdfbookmark[1]{Academic Advising}{advising}
\section*{Academic Advising}
\label{advising}
% Including academic advisement history here


% Remove \subsection{} lines and multiple tables if you only need one section!
\subsection*{Graduate}
\begin{longtable}[l]{@{}p{.125\textwidth} p{0.875\textwidth}}

    2023 & Name, University, Additional details \\

\end{longtable}

\subsection*{Undergraduate}
\begin{longtable}[l]{@{}p{.125\textwidth} p{0.875\textwidth}}

    2023 & Name, University, Additional details \\

\end{longtable}


\pdfbookmark[1]{Academic Service}{service}
\section*{Academic Service}
\label{service}


% \subsection*{Guest Editor}
% % 
% \begin{longtable}[l]{@{}p{.125\textwidth} p{0.875\textwidth}}


%     2023 & \href{https://example.com}{Link to collection} \\

% % \end{tabularx}
% \end{longtable}


% \subsection*{Journal Reviewer}
% 

% \vspace{-.75em}
\small
$*$ Independent reviewer \hspace{2em}  $\ddag$ Practicing reviewing and assisting under my advisor's supervision


\begin{longtable}[l]{@{}p{.125\textwidth} p{0.875\textwidth}}
    $*$2024 & \textit{The CRISPR Journal} \\
    $\dagger$2024 & \textit{Cell Chemical Biology} \\
    $\dagger$2024 & \textit{Nature} \\
    $\dagger$2024 & \textit{Cell Research} \\
    $\dagger$2024 & \textit{Molecular Therapy} \\
    $\dagger$2023 & \textit{ACS Synthetic Biology} \\
    $\dagger$2023 & \textit{Cell Discovery} \\
    $\dagger$2023 & \textit{National Science Review} \\
    $\dagger$2023 & \textit{Molecular Cell} \\
    $\dagger$2023 & \textit{Cell} \\
    $\dagger$2022 & \textit{Journal of Molecular Cell Biology} \\
    $\dagger$2022 & \textit{Science China Life Sciences} \\
\end{longtable}




\pdfbookmark[1]{Other Experience}{exp_other}
\section*{Other Experience}
\label{exp_other}
% Add other experience you don't know where to classify here

% \begin{longtable}[l]{@{}p{.125\textwidth} p{0.875\textwidth}}

%     2018 & Past Non-academic Job, \href{https://example.com/}{Company Name}, New York City, New York (Additional detail) \\

% \end{longtable}


% Pretty ending with the date last updated
\centering
\rule{0.25\linewidth}{0.4pt}\\
\medskip
Last updated: \today

\end{document}

% List your research experience here
\textbf{
    Feb. 2020 $\sim$ Present. Ph.D. student in  with \href{https://www.bio.pku.edu.cn/enhomes/news/teacher_dis/91.html}{Prof. Chengqi Yi}, \href{https://yilab.org.cn/}{Yi Lab}, Peking University
}

(Bioinformatics)

\begin{itemize}
    \item Independently developed mining methods based on sequence and structural analysis to 
    identify novel type VI CRISPR systems from 
    JGI metagenomic assemblies (approximately 10Tb) crawled using a 
    \href{https://github.com/hermanzhaozzzz/bioat/blob/master/src/bioat/metatools.py}{self-written Python crawler}, 
    as well as from publicly available genomes and metagenomes (approximately 5Tb) downloaded manually.
    Screening on antibiotic resistance plates of \textit{Escherichia coli} 
    yielded three novel Cas13 systems (a total of four Cas13 genes were synthesized, 
    with a success rate of 75\%).

    \item Performed off-target profiling of genome editing tools, including 
    \href{https://scholar.google.com/citations?view_op=view_citation&hl=zh-CN&user=ojSVoWQAAAAJ&citation_for_view=ojSVoWQAAAAJ:zYLM7Y9cAGgC}{CBE}\cite{lei2021detect,lei2023detect,rao2023characterizing}, 
    ABE, 
    and \href{https://scholar.google.com/citations?view_op=view_citation&hl=zh-CN&user=ojSVoWQAAAAJ&citation_for_view=ojSVoWQAAAAJ:Y0pCki6q_DkC}{DdCBE}\cite{lei2022mitochondrial,lei2023detect,rao2023characterizing}, 
    utilizing omics approaches 
    and \href{https://github.com/hermanzhaozzzz?tab=repositories&q=snakepipes&type=&language=&sort=}
    {tools that I developed}.
    \item Ph.D. thesis: ``Evaluation, Optimization, and Mechanism Studies of Cytosine Base Editors''.
    
\end{itemize}

\textbf{
    Nov. 2019 $\sim$ Jan. 2020. Lab rotation with \href{https://life.tsinghua.edu.cn/lifeen/info/1034/1077.htm}{Prof. Wei Xie}, \href{http://www.xielab.org.cn/}{Xie Lab}, Tsinghua University
}

(Bioinformatics \& Wet Experiments)

\begin{itemize}
    \item Bioinformatics analysis of ultrasensitive Ribo-seq and RNA-seq.
    \item Molecular cloning (Enzyme digestion and ligation method) and transfections.
\end{itemize}

\textbf{
    Sept. 2019 $\sim$ Oct. 2019. Lab rotation with \href{https://life.tsinghua.edu.cn/lifeen/info/1034/1083.htm}{Prof. Zhi John Lu}, \href{https://lulab.life.tsinghua.edu.cn/labhome/home/}{Lu Lab}, Tsinghua University
}

(Bioinformatics)

\begin{itemize}
    \item Bioinformatics analysis of SHAPE-seq.
    \item \href{https://scholar.google.com/citations?view_op=view_citation&hl=zh-CN&user=ojSVoWQAAAAJ&citation_for_view=ojSVoWQAAAAJ:IjCSPb-OGe4C}{RNA structure prediction}\cite{zhu2021integrative}, Motif analysis.
\end{itemize}

\textbf{
    Mar. 2019 $\sim$ Aug. 2019. Intern with \href{https://www.bio.pku.edu.cn/enhomes/news/teacher_dis/43.html}{Prof. Yulong Li}, \href{http://www.yulonglilab.org/}{Li Research Lab}, Peking University
}

(Wet Experiments)

\begin{itemize}
    \item Evolution of artificial point mutations for genetically encoded neuropeptide sensors.
    \item Cell culture, molecular cloning (Gibson homologous recombination method) and transfections.
\end{itemize}

\textbf{
    Sept. 2018 $\sim$ Feb. 2019. B.V.Sc. Thesis with \href{https://faculty.nwu.edu.cn/gaomingqing/en/index.htm}{Prof. Mingqing Gao}, Northwest A\&F University
}

(Wet Experiments)

\begin{itemize}
    \item B.V.Sc Thesis: ``The Role and Mechanism of N-3 Polyunsaturated Fatty Acids in the Inflammatory Response of Mammary Gland Epithelial Cells in Dairy Cows''.
    \item The crosstalk between the NF$\kappa$B and Nrf2 pathways.
    \item Cell culture, molecular cloning and transfections.
    \item Nuclear and cytoplasmic separation was performed, followed by western blot and immunofluorescent staining.
\end{itemize}

\textbf{
    Sept. 2016 $\sim$ Aug. 2018. Student Research Training Program with \href{https://cvm.cau.edu.cn/art/2017/9/12/art_41957_71.html}{Prof. Jian Gao}, China Agricultural University
}

(Wet Experiments)

\begin{itemize}
    \item \href{https://scholar.google.com/citations?view_op=view_citation&hl=zh-CN&user=ojSVoWQAAAAJ&citation_for_view=ojSVoWQAAAAJ:UeHWp8X0CEIC}
    {Molecular epidemiology and distribution of antimicrobial resistance genes of \textit{Staphylococcus} species isolated from Chinese dairy cows with clinical mastitis}\cite{qu2019molecular}.
    \item Bacterial typing, 16S rRNA sequencing
    \item Sanger sequencing, PCR
\end{itemize}


\textbf{
    Sept. 2015 $\sim$ June 2016. Intern with \href{https://dyxy.nwsuaf.edu.cn/en/People/FullProfessor/5e6012bf013a499c946289472b285795.htm}{Prof. Qin Zhao}. Northwest A\&F University
}

(Wet Experiments)


\begin{itemize}
    \item Entry-level research training.
    \item Molecular cloning, transfections and cell culture.
\end{itemize}
% !TEX program = xelatex

\documentclass{resume}
\usepackage{xltxtra,fontspec,xunicode}
\usepackage[slantfont,boldfont]{xeCJK} % 允许斜体和粗体
\usepackage{tabularx}
%\usepackage{zh_CN-Adobefonts_external} % Simplified Chinese Support using external fonts (./fonts/zh_CN-Adobe/)
\usepackage{zh_CN_fonts_internal} % Simplified Chinese Support using system fonts

\usepackage{titlesec}

% Spacing corresponding to `book` class
% \titlespacing*{\section} {0pt}{3.5ex plus 1ex minus .2ex}{2.3ex plus .2ex}
% \titlespacing*{\subsection} {0pt}{3.25ex plus 1ex minus .2ex}{1.5ex plus .2ex}

% Spacing before/after reduced by 1ex each
\titlespacing*{\section} {0pt}{0ex plus 1ex minus .2ex}{1.3ex plus .2ex}

\begin{document}
	\pagenumbering{gobble} % suppress displaying page number
	
	\name{赵华男(ZHAO HUANAN)}
	
	\basicInfo{
		\phone{(+86) 166-1995-4871}
		\email{hermanzhaozzzz@gmail.com}
		\faBook \hyperref[http://zhaohuanan,cc]{\underline{http://zhaohuanan.cc}}
	}

	\section{\faGraduationCap\ Education 教育经历}

	\begin{tabularx}{\textwidth}{@{}X X X r@{}}
		\textbf{清华大学} & \textbf{生命科学学院} & \textbf{生物学} & 2019.09 -- \space\space\space\space\space\space\space\space\space\space\space\space\space\space \\
		\multicolumn{4}{@{}l}{{\underline{博士研究生}} 研究方向:生物信息学,基因编辑,表观遗传学}
	\end{tabularx}

	\begin{tabularx}{\textwidth}{@{}X X X r@{}}
		\textbf{北京大学 } & \textbf{生命科学学院} & \textbf{生物学} & 2019.09 -- \space\space\space\space\space\space\space\space\space\space\space\space\space\space \\
		\multicolumn{4}{@{}l}{{\underline{博士研究生(清华)}} 联合培养}
	\end{tabularx}

	\begin{tabularx}{\textwidth}{@{}X X X r@{}}
		\textbf{西北农林科技大学} & \textbf{动医学院} & \textbf{动物医学} & 2014.09 -- 2019.06 \\
		\multicolumn{4}{@{}l}{{\underline{学士学位}} GPA: 3.7/4,国家奖学金(2015)、专业一等奖学金(2015-2019)}
	\end{tabularx}
	\begin{tabularx}{\textwidth}{@{}X X X r@{}}
		\textbf{中国农业大学} & \textbf{动医学院} & \textbf{动物医学} & 2016.09 -- 2018.06 \\
		\multicolumn{4}{@{}l}{{\underline{本科交换生}} 联合培养}
	\end{tabularx}
	\\
	
	
	
	
	
	\section{\faGraduationCap\ Scientific Experience 科研经历}
	
	\begin{tabularx}{\textwidth}{@{}X X X r@{}}
			\textbf{北京大学} & \textbf{生命科学学院} & \textbf{伊成器研究组} & 2020.02 -- \space\space\space\space\space\space\space\space\space\space\space\space\space\space \\
	\end{tabularx}

			\begin{itemize}
			 		\item 生物信息学、表观遗传学
					\item 基因编辑、Base Editor
			\end{itemize}



	\begin{tabularx}{\textwidth}{@{}X X X r@{}}
			\textbf{清华大学} & \textbf{生命科学学院} & \textbf{颉伟研究组(轮转)} & 2019.11 -- 2020.01 \\
	\end{tabularx}

			\begin{itemize}
					\item 国家重点发育生物学实验室、国家重点生物信息学实验室
					\item 生物信息学、发育生物学
					\item 转录组测序分析、Ribo-seq(转录组翻译实时测序)、转录组基序(Motif)分析
					\item 课题组网站搭建	\hyperref[http://www.xielab.org.cn]{\underline{http://www.xielab.org.cn}}
			\end{itemize}

	\begin{tabularx}{\textwidth}{@{}X X X r@{}}
			\textbf{清华大学} & \textbf{生命科学学院} & \textbf{鲁志研究组(轮转)} & 2019.08 -- 2020.10 \\
	\end{tabularx}

			\begin{itemize}
					\item 国家重点生物信息学实验室
					\item 生物信息学
					\item RNA二级结构预测、SHAPE-seq(RNA二级结构测序)、转录组基序(Motif)分析
			\end{itemize}
		
	\begin{tabularx}{\textwidth}{@{}X X X r@{}}
			\textbf{北京大学} & \textbf{生命科学学院} & \textbf{李毓龙研究组(实习)} & 2019.03 -- 2020.08 \\
	\end{tabularx}

			\begin{itemize}
					\item 麦戈文脑研究所
					\item 分子生物学、细胞生物学
					\item 神经递质探针开发
			\end{itemize}

	\begin{tabularx}{\textwidth}{@{}X X X r@{}}
			\textbf{西北农林科技大学} & \textbf{动医学院} & \textbf{张涌\&高明清研究组} & 2018.09 -- 2019.08 \\
	\end{tabularx}

			\begin{itemize}
					\item 国家重点兽医产科学实验室
					\item 毕业论文
					\item NF$\kappa$B \& Nrf2 细胞通路/信号转导
			\end{itemize}

	\begin{tabularx}{\textwidth}{@{}X X X r@{}}
			\textbf{中国农业大学} & \textbf{动医学院} & \textbf{韩博\&高健研究组} & 2017.03 -- 2018.06 \\
	\end{tabularx}

			\begin{itemize}
					\item 大学生科创项目(国家级)
					\item 分子生物学、微生物学
			\end{itemize}

		
	\section{\faGraduationCap\ Publications 发表书刊}

	\begin{itemize}
	  \item Qu Y, \textbf{Zhao H}, Nobrega D B, et al. Molecular epidemiology and distribution of antimicrobial resistance genes of Staphylococcus species isolated from Chinese dairy cows with clinical mastitis[J].
			  \textit{ \textbf{Journal of dairy science, 2019, 102(2): 1571-1583.}	}
			  \textit{小类一区 TOP} 
	  
	  \item \textbf{赵华男}, 瞿玥, 韩博, 等. 奶牛乳房炎源性产色葡萄球菌的肠毒素基因检测[J]. 
			  \textit{\textbf{Chinese Journal of Veterinary Medicine, 2018, 54(5).}}
			  \textit{中文核心} 
	\end{itemize}
	
	
	
	\section{\faGraduationCap\ Honors 荣誉}
	
	\begin{tabularx}{\textwidth}{@{}l X l r r@{}}
	&	\textbf{2019} & {清华大学} & 研究生分会  &``小研之星'' \\
	&	\textbf{2019} & {西北农林科技大学} &   &``百篇优秀毕业论文'' \\
	&	\textbf{2019} & {西北农林科技大学} &   &优秀毕业论文 \\
	&	\textbf{2019} & {西北农林科技大学} &  动医学院 &十佳优秀毕业生 \\
	&	\textbf{2019} & {西北农林科技大学} &  动医学院 & 勃林格英格翰奖学金 \\
	&	\textbf{2018} & {中国农业大学} &  科创训练(国家级) & 顺利结题 \\
	&	\textbf{2018} & {西北农林科技大学} &  创业训练(国家级) & 顺利结题 \\
	&	\textbf{2016} & {``挑战杯''} & 创业计划竞赛-校赛 & 银奖 \\
	&	\textbf{2016} & {``互联网+''} &  创新创业大赛-陕西省复赛 & 银奖 \\
	&	\textbf{2016} & {西北农林科技大学} &  化学实验技能竞赛 & 综合二等奖 \\
	&	\textbf{2015} &  &   & 国家奖学金 \\
	&	\textbf{2015-2019} & 西北农林科技大学  & 动医学院  & 专业一等奖学金 \\
	&	\textbf{2015-2019} & 西北农林科技大学  &   & 校三好学生 \\
	\end{tabularx}
%	\begin{itemize}
%  	\item Implemented xxx feature
%	\item Optimized xxx 5\%
%	\item xxx
%	\end{itemize}
	
	
	\section{\faCogs\ Skills 专业技能}
	\begin{itemize}[parsep=0.5ex]
		\item 熟练掌握动物医学/兽医理论知识与临床技能,具有良好的遗传学/细胞/分子生物学背景
		\item 熟练掌握Linux系统、版本控制工具Git及虚拟化工具Docker
		\item 熟练掌握Python/R语言,使用Jupyter-lab进行生物大数据分析
		\item 熟练掌握环境控制工具Anaconda
		\item 略懂前后端搭建(JavaScript/Django)  
		\item 具有\LaTeX 与Markdown排版语言使用经验
	\end{itemize}
	
	
%	\section{\faInfo\ More 个人评价}
%	本人硕士研究生XXX专业在读,本科XX,热爱XX,喜欢XXX相关方向的工作,具有较强的学习能力,能独立完成所承担的科研任务。熟悉xxX等计算机语言,XXX
%	
	%% Reference
	%\newpage
	%\bibliographystyle{IEEETran}
	%\bibliography{mycite}
\end{document}
